%%%%%%%%%% CUSTOM FONTS %%%%%%%%%%
% Ensemble et opérateur avec un barre précédent le symbole
\newcommand{\C}{\mathbb{C}}
\newcommand{\E}{\mathbb{E}}
\newcommand{\F}{\mathbb{F}}
\newcommand{\K}{\mathbb{K}}
\renewcommand{\P}{\mathbb{P}}
\renewcommand{\L}{\mathbb{L}}
\newcommand{\N}{\mathbb{N}}
\newcommand{\Q}{\mathbb{Q}}
\newcommand{\Z}{\mathbb{Z}}
\newcommand{\R}{\mathbb{R}}
\newcommand{\V}{\mathbb{V}}
\newcommand{\1}{\mathds{1}}
% Lettres calligaphiques
\newcommand{\cA}{\mathcal{A}}
\newcommand{\cB}{\mathcal{B}}
\newcommand{\cE}{\mathcal{E}}
\newcommand{\cF}{\mathcal{F}}
\newcommand{\cG}{\mathcal{G}}
\newcommand{\cH}{\mathcal{H}}
\newcommand{\cK}{\mathcal{K}}
\newcommand{\cU}{\mathcal{U}}
\newcommand{\cL}{\mathcal{L}}
\newcommand{\cM}{\mathcal{M}}
\newcommand{\cN}{\mathcal{N}}
\newcommand{\cC}{\mathcal{C}}
\newcommand{\cI}{\mathcal{I}}
\newcommand{\cO}{\mathcal{O}}
\newcommand{\cR}{\mathcal{R}}
\newcommand{\cS}{\mathcal{S}}
\newcommand{\cT}{\mathcal{T}}
\newcommand{\cP}{\mathcal{P}}
\newcommand{\cW}{\mathcal{W}}
% Lettres Allemandes
\newcommand{\fM}{\mathfrak{M}}
\newcommand{\fL}{\mathfrak{L}}
\newcommand{\fB}{\mathfrak{B}}
\newcommand{\fR}{\mathfrak{R}}
%%%%%%%%%%%%%%%%%%%%%%%%%%%%%%%%%%

%%%%%%%%%% ENVIRONMENTS %%%%%%%%%%
% Définition des environnements de type é noncés numérotés simplement
\newtheorem{algo}{Algorithme}
\newtheorem{theo}{Théorème}
\newtheorem{coro}{Corollaire}
\newtheorem{lemm}{Lemme}
\newtheorem{rema}{Remarque}
\newtheorem{exem}{Exemple}
\newtheorem{defi}{Définition}
\newtheorem{exer}{Exercice}
\newtheorem{prop}{Proposition}
\newtheorem{propr}{Propriétés}
\newtheorem{code}{Code}
\newtheorem{ineq}{Inégalité}
\newtheorem{prob}{Problème}
% Définition des mêmes envrionnements de type é noncés numérotés mais avec l'ajout préalable de la section
\newtheorem{adef}{Définition}[section]
\newtheorem{apro}{Proposition}[section]
\newtheorem{athe}{Théorème}[section]
\newtheorem{aexe}{Exercice}[section]
\newtheorem{aexem}{Exemple}[section]
\newtheorem{arem}{Commentaire}[section]
% Définition d'une commande qui met une boite blanche en but de ligne
\newcommand{\fdem}{\hspace*{\fill}~$\Box$\par\endtrivlist\unskip}
% Définition d'un environnement "solution d'exercice" avec un carré noir en fin de solution. La solution a une taille petite \small par rapport au texte normal. Pour le rendre plus petit remplacer par \footnotesize.
\newenvironment{solu}{\textit{Solution.\,} \footnotesize}{\hfill\ensuremath{\blacksquare}}
% Définition d'un nouvelle environnement avec un argument. La fin de l'environnement est marquè par le symbole g?n?r? par la commande \fdem
\newenvironment{proof}{\textit{Preuve.\,}}{\fdem}
%%%%%%%%%%%%%%%%%%%%%%%%%%%%%%%%%%

%%%%%%%%%% CUSTOM COMMANDS %%%%%%%%%%
% define command to mark passages needing citations
\newcommand{\tocite}{\footnote{needs citation}\marginnote{\texttt{needs citation}}}
% \todo
\newcommand{\todo}{\marginnote{\texttt{TODO}}}
% \iid
\newcommand{\iid}{\overset{\text{i.i.d}}{\sim}}
% \ind
\newcommand{\ind}{\overset{\text{ind.}}{\sim}}
% \iff % already defined
% \newcommand{\iff}{\Leftrightarrow}
% Définition d'une commande permettant de traver une ligne horizontale sur la page
\newcommand{\midline}{\begin{center}\line(1,0){0.5\textwidth}\end{center}}
% Référencement fainéant des figures
\newcommand{\tref}[1]{\textsc{Fig.} \ref{#1}}
% Indiquer la source d'une figure avec \source{La source}
\newcommand{\source}[1]{\vspace{-10pt} \caption*{\footnotesize \textbf{Source:} {#1}} }
%%%%%%%%%%%%%%%%%%%%%%%%%%%%%%%%%%%%%

%%%%%%%%%% MATH OPERATORS %%%%%%%%%%
% Bernoulli distribution
\DeclareMathOperator{\Ber}{Ber}
% \Beta
\DeclareMathOperator{\Beta}{Beta}
% Bayes Factor
\DeclareMathOperator{\BF}{BF}
% Bayesian Information Criterion
\DeclareMathOperator{\BIC}{BIC}
% Binomial distribution
\DeclareMathOperator{\Bin}{Bin}
% \Cat (categorical distribution)
\DeclareMathOperator{\Cat}{Cat}
% \Cor
\DeclareMathOperator{\Cor}{Cor}
% diagonal operator
\DeclareMathOperator{\Cov}{Cov}
% diagonal operator
\DeclareMathOperator{\diag}{diag}
% \diam
\DeclareMathOperator{\diam}{diam}
% \Dir
\DeclareMathOperator{\Dir}{Dir}
% Dirichlet process
\DeclareMathOperator{\DP}{DP}
% Kullback-Leibler divergence
\DeclareMathOperator{\KL}{KL}
% \Multi
\DeclareMathOperator{\Multi}{Multi}
% Watanabe-Akaike Information Criterion
\DeclareMathOperator{\WAIC}{WAIC}
% \Weibull
\DeclareMathOperator{\Weibull}{Weibull}
% Subject to
\DeclareMathOperator{\st}{subject~to}
%%%%%%%%%%%%%%%%%%%%%%%%%%%%%%%%%%%%