\documentclass{article}

\usepackage{lmodern}
\usepackage[utf8]{inputenc}
\usepackage[T1]{fontenc}

%%%%%%%%%%%%%%%%%%%% T E X T E %%%%%%%%%%%%%%%%%%%%%%

% package qui permettent d'utiliser tout un tas de symboles mathématiques
\usepackage{amsmath}
\numberwithin{equation}{section}

\usepackage{amssymb}
\usepackage{empheq}
\usepackage{physics}
\usepackage[squaren,Gray]{SIunits}

%ceiling function
\usepackage{mathtools}
\DeclarePairedDelimiter{\ceil}{\lceil}{\rceil}

\usepackage{csquotes}

% Polices
\usepackage{mathpazo}

% package qui permettent de personaliser les puces des listes
\usepackage{enumerate}
\usepackage{enumitem}

% package qui permettent d'introduire de la couleur
\usepackage{color}
\definecolor{mygreen}{rgb}{0,0.6,0}
\definecolor{mygray}{rgb}{0.5,0.5,0.5}
\definecolor{mymauve}{rgb}{0.58,0,0.82}

\usepackage{xspace}

% pour pouvoir barrer avec \cancel{}
\usepackage{cancel}

% pour la fonction indicatrice 1
\usepackage{dsfont}

% pour utiliser le symbole euro par \euro
\usepackage{eurosym}

% afficher des nombres en mode texte
\usepackage{numprint}

% to insert notes in the margin. Can be useful during the editorial process
\usepackage{marginnote}
\usepackage[top=1.5cm, bottom=1.5cm, outer=5cm, inner=2cm, heightrounded, marginparwidth=2.5cm, marginparsep=1cm]{geometry}


%%%%%%%%%%%%%%%%%%%% F I G U R E S %%%%%%%%%%%%%%%%%%%%%%

% package qui permettent d'inclure des graphiques
%\usepackage{graphicx}
%\numberwithin{figure}{section}

%Pour utiliser des sous-figures
%\usepackage{subcaption}

%Pour utiliser [H] pour le placement des figures
%\usepackage{float}

%figure entourée de texte
%\usepackage{wrapfig}
% \begin{wrapfigure}{L}{0.4\textwidth}

%Keep figures in section
%\usepackage[section]{placeins}
% \FloatBarrier % stops floats from descending further

% Pour des références stylées
%\usepackage{varioref}
%\usepackage[german]{fancyref} % dispo seulement en allemand
\usepackage[linkcolor=black,
						urlcolor=black,
						colorlinks=true,
						citecolor=mygreen]{hyperref}% package qui créer des liens dynamiques dans un fichier pdf à partir de tout élément taggé
%\usepackage{cleveref}


%%%%%%%%%%%%%%%%%%%% D E S S I N %%%%%%%%%%%%%%%%%%%%%%

%pour dessiner des graphes de fonctions
%\usepackage{pgfplots}

%%%%%%%%%%%%%%%%%%%% H E A D E R et F O O T E R %%%%%%%%%%%%%%%%%%%%%%

% \usepackage{fancyhdr}
% \fancyfoot{}
% \fancyfoot[LE,RO]{\thepage}


%%%%%%%%%%%%%%%%%%%%%% D I V E R S %%%%%%%%%%%%%%%%%%%%%%%%%%

%Bibliographie
\usepackage[backend=bibtex,sorting=ynt]{biblatex}%sorting=none pour avoir les références selon d'ordre d'apparition dans le texte
% sorting=ynt		backend=biber,		citestyle=alphabetic		url=false

% Set a watermark
% \usepackage{draftwatermark}
% \SetWatermarkText{© Till \textsc{Kletti} et al.}
% \SetWatermarkFontSize{1cm}
% \SetWatermarkScale{1}
% \SetWatermarkLightness{1} % white for invisibility
% \SetWatermarkAngle{0}

%%%%%%%%%%%%%%%%%%% C O D E %%%%%%%%%%%%%%%%%%%%%%%

% Pour écrire du pseudo-code
\usepackage{algpseudocode}
\usepackage{algorithm}
\algdef{SE}[DOWHILE]{Do}{doWhile}{\algorithmicdo}[1]{\algorithmicwhile\ #1}

% pour insérer du code
\usepackage{listings}
\usepackage{listingsutf8}
% sur une page Fige les paramètres de hauteur et largeur du texte, de hauteur verticale au haut de page, de longueur horizontale au bord gauche de page de No pair, impair 
\setlength{\textheight}{24cm}
\setlength{\textwidth}{16cm}

\setlength{\topmargin}{-2cm}

\setlength{\evensidemargin}{0cm}
\setlength{\oddsidemargin}{0cm}

% Redefinit l'espace entre les lignes d'un paragraphe. Valur par défaut 1
%\renewcommand{\baselinestretch}{1.2}

% Paramètre l'espace après un paragraphe
%\parskip = 3mm
% Paramètre l'espace d'indentation avant un début de paragraphe
%\parindent 5mm
% Positionne par rapport à la marge le premier niveau d'item d'une liste
%\leftmargini 1.3em

%%%%%%%%%% CUSTOM FONTS %%%%%%%%%%
% Ensemble et opérateur avec un barre précédent le symbole
\newcommand{\C}{\mathbb{C}}
\newcommand{\E}{\mathbb{E}}
\newcommand{\F}{\mathbb{F}}
\newcommand{\K}{\mathbb{K}}
\renewcommand{\P}{\mathbb{P}}
\renewcommand{\L}{\mathbb{L}}
\newcommand{\N}{\mathbb{N}}
\newcommand{\Q}{\mathbb{Q}}
\newcommand{\Z}{\mathbb{Z}}
\newcommand{\R}{\mathbb{R}}
\newcommand{\V}{\mathbb{V}}
\newcommand{\1}{\mathds{1}}
% Lettres calligaphiques
\newcommand{\cA}{\mathcal{A}}
\newcommand{\cB}{\mathcal{B}}
\newcommand{\cE}{\mathcal{E}}
\newcommand{\cF}{\mathcal{F}}
\newcommand{\cG}{\mathcal{G}}
\newcommand{\cH}{\mathcal{H}}
\newcommand{\cK}{\mathcal{K}}
\newcommand{\cU}{\mathcal{U}}
\newcommand{\cL}{\mathcal{L}}
\newcommand{\cM}{\mathcal{M}}
\newcommand{\cN}{\mathcal{N}}
\newcommand{\cC}{\mathcal{C}}
\newcommand{\cI}{\mathcal{I}}
\newcommand{\cO}{\mathcal{O}}
\newcommand{\cR}{\mathcal{R}}
\newcommand{\cS}{\mathcal{S}}
\newcommand{\cT}{\mathcal{T}}
\newcommand{\cP}{\mathcal{P}}
\newcommand{\cW}{\mathcal{W}}
% Lettres Allemandes
\newcommand{\fM}{\mathfrak{M}}
\newcommand{\fL}{\mathfrak{L}}
\newcommand{\fB}{\mathfrak{B}}
\newcommand{\fR}{\mathfrak{R}}
%%%%%%%%%%%%%%%%%%%%%%%%%%%%%%%%%%

%%%%%%%%%% ENVIRONMENTS %%%%%%%%%%
% Définition des environnements de type é noncés numérotés simplement
\newtheorem{algo}{Algorithme}
\newtheorem{theo}{Théorème}
\newtheorem{coro}{Corollaire}
\newtheorem{lemm}{Lemme}
\newtheorem{rema}{Remarque}
\newtheorem{exem}{Exemple}
\newtheorem{defi}{Définition}
\newtheorem{exer}{Exercice}
\newtheorem{prop}{Proposition}
\newtheorem{propr}{Propriétés}
\newtheorem{code}{Code}
\newtheorem{ineq}{Inégalité}
\newtheorem{prob}{Problème}
% Définition des mèmes envrionnements de type é noncés numérotés mais avec l'ajout préalable de la section
\newtheorem{adef}{Définition}[section]
\newtheorem{apro}{Proposition}[section]
\newtheorem{athe}{Théorème}[section]
\newtheorem{aexe}{Exercice}[section]
\newtheorem{aexem}{Exemple}[section]
\newtheorem{arem}{Commentaire}[section]
% Définition d'une commande qui met une boite blanche en but de ligne
\newcommand{\fdem}{\hspace*{\fill}~$\Box$\par\endtrivlist\unskip}
% Définition d'un environnement "solution d'exercice" avec un carré noir en fin de solution. La solution a une taille petite \small par rapport au texte normal. Pour le rendre plus petit remplacer par \footnotesize.
\newenvironment{solu}{\textit{Solution.\,} \footnotesize}{\hfill\ensuremath{\blacksquare}}
% Définition d'un nouvelle environnement avec un argument. La fin de l'environnement est marquè par le symbole g?n?r? par la commande \fdem
\newenvironment{proof}{\textit{Preuve.\,}}{\fdem}
%%%%%%%%%%%%%%%%%%%%%%%%%%%%%%%%%%

%%%%%%%%%% CUSTOM COMMANDS %%%%%%%%%%
% define command to mark passages needing citations
\newcommand{\tocite}{\footnote{needs citation}\marginnote{\texttt{needs citation}}}
% \todo
\newcommand{\todo}{\marginnote{\texttt{TODO}}}
% \iid
\newcommand{\iid}{\overset{\text{i.i.d}}{\sim}}
% \ind
\newcommand{\ind}{\overset{\text{ind.}}{\sim}}
% \iff % already defined
% \newcommand{\iff}{\Leftrightarrow}
% Définition d'une commande permettant de traver une ligne horizontale sur la page
\newcommand{\midline}{\begin{center}\line(1,0){0.5\textwidth}\end{center}}
% Référencement fainéant des figures
\newcommand{\tref}[1]{\textsc{Fig.} \ref{#1}}
% Indiquer la source d'une figure avec \source{La source}
\newcommand{\source}[1]{\vspace{-10pt} \caption*{\footnotesize \textbf{Source:} {#1}} }
%%%%%%%%%%%%%%%%%%%%%%%%%%%%%%%%%%%%%

%%%%%%%%%% MATH OPERATORS %%%%%%%%%%
% Bernoulli distribution
\DeclareMathOperator{\Ber}{Ber}
% \Beta
\DeclareMathOperator{\Beta}{Beta}
% Bayes Factor
\DeclareMathOperator{\BF}{BF}
% Bayesian Information Criterion
\DeclareMathOperator{\BIC}{BIC}
% Binomial distribution
\DeclareMathOperator{\Bin}{Bin}
% \Cat (categorical distribution)
\DeclareMathOperator{\Cat}{Cat}
% \Cor
\DeclareMathOperator{\Cor}{Cor}
% diagonal operator
\DeclareMathOperator{\Cov}{Cov}
% diagonal operator
\DeclareMathOperator{\diag}{diag}
% \diam
\DeclareMathOperator{\diam}{diam}
% \Dir
\DeclareMathOperator{\Dir}{Dir}
% Dirichlet process
\DeclareMathOperator{\DP}{DP}
% Kullback-Leibler divergence
\DeclareMathOperator{\KL}{KL}
% \Multi
\DeclareMathOperator{\Multi}{Multi}
% Watanabe-Akaike Information Criterion
\DeclareMathOperator{\WAIC}{WAIC}
% \Weibull
\DeclareMathOperator{\Weibull}{Weibull}
%%%%%%%%%%%%%%%%%%%%%%%%%%%%%%%%%%%%

\addbibresource{Biblio}

\title{User Equilibrium Solution}
\author{Zheng Li}

\begin{document}

\maketitle

% \begin{abstract}
%     Traffic flow assignment..
% \end{abstract}

\section{Problem Statement}

\subsection{Decision Variables and Parameters}

Suppose we have a road network, which can be represented by nodes and directed links. We denote the nodes set as $ \cN $, and the links set $ \cA $, whose elements are ordered tuples of form $ (n_1, n_2) $, where $ n_1, n_2 \in \cN $. We also have set of origin nodes $ \cR \subset \cN $ and set of destination nodes $ \cS \subset \cN $. Then, we shall define the Origin-Destination (O-D) pair, which is exactly the ordered tuple of form $ (r, s) $, where $ r \in \cR $ and $ s \in \cS $.

Moreover, we need the concept of path. We denote the set of paths connecting O-D pair $ r $-$ s $ as $ \cK_{rs} $, whose element can be roughly represented as
$$ \{ (r, n_1), (n_1, n_2), \cdots, (n_{k-1}, n_k), (n_k, s) \} $$ 
where $ (r, n_1) \in \cA $, $ (n_k, s) \in \cA $, and $ (n_j, n_{j+1}) \in \cA $ for all $ 2 \le j < k $; $ n_1, n_2, \cdots, n_k \in \cN $, $ r \in \cR $ and $ s \in \cS $; $ k \in \N $. In particular, we shall introduce function $ \delta_{a, k}^{rs} $ to represent the relationship between link and path; that is, if the link $ a \in \cA $ belongs to $ k \in \cK_{rs} $, then $ \delta_{a, k}^{rs} = 1 $, otherwise it equals to $ 0 $.

Now, we should introduce the concept of flow, which represents the number of passenger car units passing through specific link or path in some unit time. We denote the flow on the link $ a \in \cA $ as $ x_a $, denote $ f_k^{rs} $ as the flow on path $ k $ connecting O-D pair $ r $-$ s $. Notice that $ f_k^{rs} $ are exactly our decision variables, and $ x_a, f_k^{rs} \in \R^+ := [0, + \infty) $. On the other hand, for each O-D pair $ r $-$ s $, there is a conjugated term: demand $ q_{rs} $, which indicates the traffic demand between origin $ r $ and destination $ s $ during the observation period. Thus, we apparently have the constraint: 
$$ \sum_{k \in \cK_{rs}} f_{k}^{rs} = q_{rs} ~~~~ \forall r \in \cR ~~ \forall s \in \cS $$
Beside, we can also use $ \delta_{a, k}^{rs} $ to show the relationship between link flow $ x_a $ and path flow $ f_k^{rs} $, which can be written as: 
\begin{equation} \label{eq-path-flow-to-link-flow}
    x_a = \sum_{r \in \cR} \sum_{s \in \cS} \sum_{k \in \cK_{rs}} f_{k}^{rs} \delta_{a, k}^{rs} ~~~~ \forall a \in \cA
\end{equation}

Associated with each link $ a \in \cA $, there is link performance function $ p_a : \R^+ \rightarrow \R^+ $, which express the traveling time for each car unit under given traffic flow, with the following assumptions:
\begin{enumerate}
    \item $ p \in \cC^1 (\R^+) $, i.e. $ f $ is continuously differentiable on $ \R^+ $.
    \item $ p $ is strictly increasing on $ \R^+ $.
\end{enumerate}
The traveling time on the path $ k \in \cK_{rs} $ is denoted as $ c_k^{rs} $, which can be written as: 
\begin{equation} \label{eq-link-time-to-path-time}
    c_k^{rs} = \sum_a t_a \delta_{a, k}^{rs} ~~~~ \forall k \in \cK_{rs} ~~ \forall r \in \cR ~~ \forall s \in \cS
\end{equation}
where $ t_a := p_a (x_a) $ is the traveling time on the link $ a \in \cA $.

\subsection{Objective and Definition of User Equilibrium}

The definition of user equilibrium can be concisely expressed as \cite{Shef85}: no traveler can improve his travel time by unilaterally changing routes.

Meanwhile, Wardrop \cite{wardrop1952road} has provided an equivalent definition of user equilibrium, which is call \textit{Wardrop's First Principle}: For each O-D pair, at user equilibrium, the travel time on all used paths is equal, and (also) less than or equal to the travel time that would be experienced by a single vehicle on any unused path. Therefore, we can state our problem in the following way:

\begin{prob} \label{prob-wardrop}
    Given road network $ \{ \cN, \cE, \cR, \cS \} $, traffic demands $ (q_{rs})_{r \in \cR, s \in \cS} $ and link performance functions $ (p_a)_{a \in \cA} $. Determine all $ f_k^{rs} \in \R^+ $ such that the \textit{Wardrop First Principle} holds.
\end{prob}

In other words, for determined flow on paths $ \boldsymbol{f} := \left( f_k^{rs} \right)_{k \in \cK^{rs}}^{r \in \cR, s \in \cS} $, by Equality \ref{eq-path-flow-to-link-flow} we can calculate flow on links $ \boldsymbol{x} := (x_a)_{a \in \cA} $. Then, using link performance functions $ (p_a)_{a \in \cA} $ we can obtain the traveling time of links $ (t_a)_{a \in \cA} $. According to Equality \ref{eq-link-time-to-path-time} we can compute the traveling time of paths $ \left( c_k^{rs} \right)_{k \in \cK^{rs}}^{r \in \cR, s \in \cS} $. As we said, the result $ \left( c_k^{rs} \right)_{k \in \cK^{rs}}^{r \in \cR, s \in \cS} $ must satisfies the \textit{Wardrop First Principle}, that is: for each fixed $ r \in \cR $ and $ s \in \cS $, there exists a constant $ c_{rs} \in \R^+ $ such that: $ c_k^{rs} \equiv c_{rs} $ for all $ k \in K_{rs} $ such that $ f_k^{rs} > 0 $; and $ c_k^{rs} \ge c_{rs} $ for all $ k \in \cK_{rs} $ such that $ f_k^{rs} = 0 $. 

\section{Equivalent Mathematical Formulation}

\subsection{Statement of Equivalent Formulation}

The most famous mathematical formulation for the user equilibrium is as follows:

\begin{prob} \label{prob-min-formulation}
    Find the minimizer, if it exists, of the optimization problem
    \begin{align} \label{eq-opt-objective-func}
        \min_{\boldsymbol{f} \ge \boldsymbol{0}}~ z(\boldsymbol{f}) &= \sum_{a \in \cA} \int_0^{x_a (\boldsymbol{f})} p_a (\omega) d \omega \\ 
        \sum_{k \in \cK_{rs}} f_{k}^{rs} &= q_{rs} ~~~~ \forall r \in \cR ~~ \forall s \in \cS \label{eq-opt-constraint-last}
    \end{align}
    in which 
    $$ x_a (\boldsymbol{f}) = \sum_{r \in \cR} \sum_{s \in \cS} \sum_{k \in \cK_{rs}} f_{k}^{rs} \delta_{a, k}^{rs} $$
    for all $ a \in \cA $, according to Equality \ref{eq-path-flow-to-link-flow}.
\end{prob}

We shall notice that the objective function does not have any intuitive interpretation.

\subsection{Existence of Minimizer}
 
Let us denote $ N := \sum_{r \in \cR} \sum_{s \in \cS} \abs{\cK_{rs}} $, $ M := \abs{\cE} $, and the the feasible set of Problem \ref{eq-opt-objective-func} - \ref{eq-opt-constraint-last} as $ F \subset \R^N $, which exactly is
$$ F = \{ \boldsymbol{f} \in \R^N : \boldsymbol{f} \ge 0; \sum_{k \in \cK_{rs}} f_{k}^{rs} = q_{rs}, ~\forall r \in \cR, ~\forall s \in \cS \} $$
Obviously $ F $ is a closed subset of $ \R^N $. 

Moreover, notice that the linear (therefore continuous) map $ \boldsymbol{x} : F \subset \R^N \rightarrow \R^M $ can be represented as a matrix with binary entries, i.e. matrix contains only zero or one, but it cannot be zero matrix, otherwise it will be meaningless. This property tells us: the non-negativity of $ \boldsymbol{x} $ can be inherited from the non-negativity of $ \boldsymbol{f} $. We affirm the properties of $ z : \R^N \rightarrow \R $ as:
\begin{enumerate}
    \item $ z $ is coercive with respect to $ \boldsymbol{f} $. In fact, if $ \boldsymbol{f} \rightarrow +\infty $, then there exists $ a_0 \in \cA $ such that $ x_{a_0} (\boldsymbol{f}) \rightarrow + \infty $; by non-negativity, monotonicity and continuity of $ p_a $, 
    $$ z (\boldsymbol{f}) = \sum_{a \in \cA} \int_0^{x_a (\boldsymbol{f})} p_a (\omega) d \omega \ge \int_0^{x_{a_0} (\boldsymbol{f})} p_a (\omega) d \omega \rightarrow \infty $$ 
    \item $ z \in \cC^0 (F) $, as the composition of continuous map $ \boldsymbol{x} \mapsto \sum_{a \in \cA} \int_0^{x_a} p_a (\omega) d \omega $ and linear map $ x_a $.
\end{enumerate}

The coercivity of $ z $ tells us the minimizer cannot appear at the infinity, then no matter where the minimizer is (if it exists), we can constrict it in a bounded and closed (since the feasible set is closed) set $ F_0 \subset F $; in other words, in a compact set $ F_0 $. The continuity of $ z $ implies the lower semi-continuity. 

\begin{theo} [Weierstrass] \label{theo-weierstrass}
    Suppose $ F_0 \subset \R^N $ is a compact set, and $ z : F_0 \rightarrow \R $ is lower semi-continuous, then there exists $ \tilde{f} \in F_0 $ such that 
    $$ z(\tilde{f}) = \min_{f \in F_0} z(f) $$
\end{theo}

Theorem \ref{theo-weierstrass} shows the existence of minimizer of Problem \ref{eq-opt-objective-func} - \ref{eq-opt-constraint-last}.

\subsection{Convexity of Equivalent Formulation}

Let us recall the definition of convex optimization problem:

\begin{defi}[Convex optimization Problem]
    A convex optimization problem is an optimization problem in which the objective function is a convex function and the feasible set is a convex set.
\end{defi}

As we know, in convex optimization problems, every local minimum is a global minimum; moreover, if the objective function is strictly convex, then we have also the uniqueness of minimizer. However, convex optimization problem does not always admit the minimizer; but luckily in the last section we have already shown the existence of minimizer of Problem \ref{eq-opt-objective-func} - \ref{eq-opt-constraint-last}.

Observe that, choose arbitrary $ \boldsymbol{f}_1, \boldsymbol{f}_2 \in F $, their convex combination $ t \boldsymbol{f}_1 + (1-t) \boldsymbol{f}_2 \in F $, where $ t \in [0, 1] $, thus $ F $ is convex. For demonstrating the convexity of $ z $, we need the following lemma:

\begin{lemm} \label{lemm-ineq-int}
    Suppose $ p : \R^+ \rightarrow \R^+ $ is differentiable and strictly increasing, then for given $ x > 0 $, we have
    $$ \int_0^{tx} p (\omega) d \omega < t \int_0^x p (\omega) d \omega $$
    for all $ t \in (0, 1) $.
\end{lemm}
\begin{proof}
    Define 
    $$ y (t) := \int_0^{tx} p (\omega) d \omega - t \int_0^x p (\omega) d \omega $$
    where $ t \in [0, 1] $. Obviously $ y (0) = y (1) = 0 $, and we have $ y^{\prime\prime}(t) = p^\prime(tx) \cdot x^2 $, which tells us $ y^{\prime\prime}(t) \ge 0 $ for $ t > 0 $ on $ [0, 1] $ since $ x > 0 $ and $ p $ is strictly increasing on $ \R^+ $. Therefore $ y(t) < 0 $ on $ (0, 1) $, the inequality holds.
\end{proof}

Now, choose arbitrary $ \boldsymbol{f}_1, \boldsymbol{f}_2 \in F $ such that $ \boldsymbol{f}_1 \neq \boldsymbol{f}_2 $, and let $ t \in (0, 1) $. With the help of Lemma \ref{lemm-ineq-int}, there is
\begin{align}
    z (t \boldsymbol{f}_1 + (1-t) \boldsymbol{f}_2) &= \sum_{a \in \cA} \int_0^{x_a (t \boldsymbol{f}_1 + (1-t) \boldsymbol{f}_2)} p_a (\omega) d \omega \\ 
    &= \sum_{a \in \cA} \int_0^{t x_a (\boldsymbol{f}_1) + (1-t) x_a (\boldsymbol{f}_2)} p_a (\omega) d \omega \\
    &= \sum_{a \in \cA} \int_0^{t x_a (\boldsymbol{f}_1)} p_a (\omega) d \omega + \sum_{a \in \cA} \int_0^{(1-t) x_a (\boldsymbol{f}_2)} p_a (\omega) d \omega \\
    &<  t \sum_{a \in \cA} \int_0^{x_a (\boldsymbol{f}_1)} p_a (\omega) d \omega + (1 - t) \sum_{a \in \cA} \int_0^{x_a (\boldsymbol{f}_2)} p_a (\omega) d \omega \\ 
    &= t z(\boldsymbol{f}_1) + (1 - t) z (\boldsymbol{f}_2)
\end{align}
Notice that we have stated that $ \boldsymbol{x} \ge 0 $.

We have proved $ F $ is convex and $ z $ is strictly convex. Therefore, the Problem \ref{eq-opt-objective-func}-\ref{eq-opt-constraint-last} is convex optimization problem and it admits a unique minimizer.

\subsection{Review on Constrained Problems}

We have to refresh ourselves with an important result in constrained optimization problem. Consider
\begin{align} \label{pb-obj-function}
    \min_{x \in X \subset \R^n}~ f(x) & \\ 
    \st~ h_i (x) &= 0 ~~~~ \forall i \in I := \{1, 2, \cdots, m\} \\
    g_j (x) &\le 0 ~~~~ \forall j \in J := \{1, 2, \cdots, p\} \label{pb-last-constraint}
\end{align}
where $ m \le n $ and the functions $ f, h_i, g_j \in \cC^1(X) $ for all $ i \in I $ and for all $ j \in J $. 

Before introducing the important results, we have to define:

\begin{defi} [Regular Point \cite{luenberger2008linear}] 
    Let $ x^* \in X $ be a point satisfying the constraints $ h_i (x^*) = 0 $ for all $ i \in I $ and $ g(x^*) \le 0 $ for all $ j \in J $. Moreover, denote $ J_a \subset J $ as the set of indices for which $ g_j (x^*) = 0 $, that is, active constraint. Then $ x^* $ is said to be a regular point if the gradient vectors $ \nabla h_i (x^*) $ and $ \nabla g_j (x^*) $, for all $ i \in I $ and for all $ j \in J_a $, are linearly independent.
\end{defi}

Then we have the following results:

\begin{theo}[Karush-Kuhn-Tucker Conditions \cite{luenberger2008linear}] Let $ x^* $ be a local minimum point for the Problem \ref{pb-obj-function} - \ref{pb-last-constraint}, and suppose $ x^* $ is a regular point for the constraints. Then there is a vector $ \lambda \in \R^m $ and a vector $ \mu \in \R ^p $ with $ \mu \ge 0 $ such that
\begin{align*}
    \nabla f (x^*) + \sum_{i \in I} \lambda_i \nabla h_i (x^*) + \sum_{j \in J} \mu_j \nabla g_j (x^*) &= 0 \\ 
    \mu_j g_j(x^*) &= 0 ~~~~ \forall j \in J
\end{align*}
\end{theo}

\begin{theo} [KKT Conditions for Convex Optimization Problems] \label{theo-kkt-for-convex}
    $ x^* \in X $ is an optimal solution if and only if $ x^* $ satisfies the KKT conditions.
\end{theo}

Above introduced theorems will play vital role in the proof of equivalence between Problem \ref{prob-wardrop} and Problem \ref{prob-min-formulation} in the next section.

\subsection{Demonstration of Equivalence}

At the beginning we need to do some computation. By Equality \ref{eq-path-flow-to-link-flow}, we can derive
\begin{equation} \label{eq-derivative-x-f}
    \frac{\partial x_a (\boldsymbol{f})}{\partial f^{rs}_k} = \delta^{rs}_{a, k} ~~~~ \forall k \in \cK_{rs} ~~ \forall r \in \cR ~~ \forall s \in \cS
\end{equation}
Thanks to Equality \ref{eq-derivative-x-f}, directly we get
\begin{equation} \label{eq-derivative-z-f}
    \frac{\partial z (\boldsymbol{f})}{\partial f_k^{rs}} = \sum_{a \in \cA} \left[ p_a \left(x_a (\boldsymbol{f})\right) \frac{\partial x_a (\boldsymbol{f})}{\partial f_k^{rs}} \right] = \sum_{a \in \cA} \left[ p_a \left(x_a (\boldsymbol{f})\right) \delta_{a, k}^{rs} \right] = c_k^{rs} (\boldsymbol{f}) ~~~~ \forall k \in \cK_{rs} ~~ \forall r \in \cR ~~ \forall s \in \cS
\end{equation}

Moreover, for the accordance, we transform the constraints in Problem \ref{eq-opt-objective-func}-\ref{eq-opt-constraint-last} as
\begin{equation}
    h_{rs} (\boldsymbol{f}) = \sum_{k \in \cK_{rs}} f_k^{rs} - q_{rs} = 0 ~~~~ \forall r \in \cR ~~ \forall s \in \cS 
\end{equation}
Similarly, 
\begin{equation}
    g^{rs}_k (\boldsymbol{f}) = - f_{rs}^k \le 0 ~~~~ \forall k \in \cK_{rs} ~~ \forall r \in \cR ~~ \forall s \in \cS
\end{equation}

Obviously, we have
\begin{equation} \label{eq-derivative-h-f}
    \frac{\partial h_{rs} (\boldsymbol{f})}{\partial f_{k}^{r's'}} = \1_{\{ r = r', ~s = s' \}} ~~~~ \forall r' \in \cR ~~ \forall s' \in \cS
\end{equation}
and
\begin{equation} \label{eq-derivative-g-f}
    \frac{\partial g_k^{rs} (\boldsymbol{f})}{\partial f_k^{rs}} = - \1_{\{r=r', ~s=s', ~k=k'\}} ~~~ \forall k' \in \cK_{r's'} ~~ \forall r' \in \cR ~~ \forall s' \in \cS
\end{equation}
in which $ \1_X (x) $ is indicator function: $ \1_X (x) = 1 $ if $ x \in X $, $ \1_X (x) = 0 $ if $ x \notin X $. Equalities \ref{eq-derivative-h-f} and \ref{eq-derivative-g-f} also tell us all $ f \in F $ are regular points.

Now we suppose that $ \tilde{\boldsymbol{f}} $ is a global minimum, and we know it must exist. By Theorem \ref{theo-kkt-for-convex}, the global minimizer is characterized by the KKT conditions. Apply the KKT conditions to our Problem \ref{eq-opt-objective-func} - \ref{eq-opt-constraint-last}, we know there exist $ \lambda_{rs} \in \R $ for all $ r \in \cR $, and for all $ s \in \cS $; and $ \mu^{rs}_k \in \R^+ $ for all $ k \in \cK_{rs} $, for all $ r \in \cR $, and for all $ s \in \cS $; such that 
\begin{align}
    &\nabla z (\tilde{\boldsymbol{f}}) + \sum_{r \in \cR} \sum_{s \in \cS} \lambda_{rs} \nabla h_{rs} (\tilde{\boldsymbol{f}}) + \sum_{r \in \cR} \sum_{s \in \cS} \sum_{k \in \cK_{rs}} \mu_k^{rs} \nabla g_k^{rs} (\tilde{\boldsymbol{f}}) = 0 \\ 
    &\mu_k^{rs} g_k^{rs} (\tilde{\boldsymbol{f}}) = 0 ~~~~ \forall k \in \cK_{rs} ~~ \forall r \in \cR ~~ \forall s \in \cS
\end{align}
Combined with Equalities \ref{eq-derivative-z-f}, \ref{eq-derivative-h-f} and \ref{eq-derivative-g-f}, there are
\begin{align}
    c_k^{rs} (\tilde{\boldsymbol{f}}) &= - \lambda_{rs} + \mu_k^{rs} ~~~~ \forall k \in \cK_{rs} ~~ \forall r \in \cR ~~ \forall s \in \cS \\ 
    \mu_k^{rs} \tilde{\boldsymbol{f}}^{rs}_k &= 0 ~~~~ \forall k \in \cK_{rs} ~~ \forall r \in \cR ~~ \forall s \in \cS
\end{align}
which implies: for fixed $ r \in \cR $ and $ s \in \cS $, for $ k \in \cK_{rs} $ such that $ \tilde{\boldsymbol{f}}^{rs}_k \neq 0 $, there must be $ \mu_k^{rs} = 0 $, then $ \tilde{\boldsymbol{f}}^{rs}_k \equiv - \lambda_{rs} $; for $ k \in \cK_{rs} $ such that $ \tilde{\boldsymbol{f}}^{rs}_k = 0 $, since $ \mu_k^{rs} \ge 0 $, then $ \tilde{\boldsymbol{f}}^{rs}_k \ge - \lambda_{rs} $. Notice this is exactly the characterization of user equilibrium solution, thus we have complete the demonstration of equivalence between the Problem \ref{prob-wardrop} and Problem \ref{prob-min-formulation}.

\section{Introduction to Frank-Wolfe Algorithm}

\printbibliography

\end{document}